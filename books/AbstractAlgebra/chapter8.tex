\documentclass[12pt]{article}

\usepackage{setspace}
\usepackage[margin=1in]{geometry}
\usepackage{amsmath,amsthm,amssymb,amsfonts}

% Create the margins required for class %
\oddsidemargin=-0.5cm                 	%
\setlength{\textwidth}{6.5in}         	%
\addtolength{\voffset}{-20pt}        		%
\addtolength{\headsep}{25pt}           	%

% Tells LaTeX to allow you to enter information in the heading
\pagestyle{myheadings}
\markright{Yetong Li\hfill \today \hfill}

% Use double spacing throughout the document
% \doublespacing

% Define commonly used sets
\newcommand{\N}{\mathbb{N}} % Natural numbers
\newcommand{\Z}{\mathbb{Z}} % Integers
\newcommand{\R}{\mathbb{R}} % Real numbers
\newcommand{\Q}{\mathbb{Q}} % Rational numbers
\newcommand{\C}{\mathbb{C}} % Complex numbers

% Define a bunch of theorem-like environments
\newtheorem{theorem}{Theorem}
\newtheorem{lemma}{Lemma}
\newtheorem{problem}{Problem}
\newtheorem{question}{Question}
\newtheorem{corollary}{Corollary}
\newtheorem{proposition}{Proposition}

\begin{document}

% -------------------------------------------------------------- %
%                         Starts here                            %
% -------------------------------------------------------------- %

\title{Chapter 8: Permutations of A Finite Set}
\author{Yetong Li \\ A Book of Abstract Algebra}

\maketitle


\begin{lemma}
  For two transpositions, $(xa)$ and $(xb)$, where $a \neq b$,
  prove that $(xa)(xb) = (xa)(ab)$.
  \begin{proof}
    Consider when $z$ is applied to both sides:
    \begin{align*}
      (xa)(xb)(z) & = \begin{cases}
                        z & \text{if } z \neq x, a, b \\
                        a & \text{if } z = x          \\
                        b & \text{if } z = a          \\
                        x & \text{if } z = b
                      \end{cases} \\
    \end{align*}
    On the right-hand side, we have:
    \begin{align*}
      (xa)(ab)(z) & = \begin{cases}
                        z & \text{if } z \neq x, a, b \\
                        a & \text{if } z = x          \\
                        b & \text{if } z = a          \\
                        x & \text{if } z = b
                      \end{cases} \\
    \end{align*}
    Since both sides yield the same result for all $z$, we conclude that:
    \[
      (xa)(xb) = (xa)(ab)
    \]
  \end{proof}
\end{lemma}


\begin{lemma}
  Let $\pi \in S_n$. If $\pi$ is a product of $k$ transpositions, then
  $\pi^{-1}$ is also a product of $k$ transpositions.
  \begin{proof}
    We can express $\pi$ as a product of transpositions:
    \[
      \pi = (x_1 y_1)(x_2 y_2) \cdots (x_k y_k)
    \]
    where each $(x_i y_i)$ is a transposition. The inverse of $\pi$ is given by:
    \[
      \pi^{-1} = (x_k y_k)^{-1} \cdots (x_2 y_2)^{-1} (x_1 y_1)^{-1}
    \]
    Since the inverse of a transposition is itself, we have:
    \[
      \pi^{-1} = (y_k x_k)(y_{k-1} x_{k-1}) \cdots (y_1 x_1)
    \]
    This shows that $\pi^{-1}$ is also a product of $k$ transpositions,
    specifically the same transpositions but in reverse order. Thus, if $\pi$ is a
    product of $k$ transpositions, then $\pi^{-1}$ is also a product of $k$ transpositions.
  \end{proof}
\end{lemma}


\begin{lemma}
  The set of all even permutations in $S_n$ forms a subgroup of $S_n$.
  \begin{proof}
    Let $A_n$ denote the set of all even permutations in $S_n$.
    To show that $A_n$ is a subgroup, we need to show that it is closed under multiplication and inverses.
    \begin{itemize}
      \item \textbf{Closure under multiplication:} If $\pi, \sigma \in A_n$,
            then both $\pi$ and $\sigma$ can be expressed as products of an even number of transpositions.
            The product $\pi \sigma$ can be expressed as a product of the transpositions of $\pi$ followed by those of $\sigma$.
            Since the sum of two even numbers is even, $\pi \sigma$ is also a product of an even number of transpositions, hence $\pi \sigma \in A_n$.
      \item \textbf{Closure under inverses:} If $\pi \in A_n$, then $\pi$ can be expressed as a product of an even number of transpositions.
            By \textbf{Lemma 2}, the inverse $\pi^{-1}$ is also a product of the same transpositions in reverse order.
            Since the number of transpositions remains even, $\pi^{-1} \in A_n$.
    \end{itemize}
    Since $A_n$ is closed under multiplication and inverses, it is a subgroup of $S_n$.
  \end{proof}
\end{lemma}


\begin{proposition}[Exercise B.3]
  Let $\alpha$ be a cycle of length $s$, say $\alpha = (a_1 a_2 \ldots a_s)$.
  Show that $\alpha^{-1} = \alpha^{s-1}$.
  \begin{proof}
    Let $a_i \in \{a_1, a_2, \cdots , a_s\}$.
    We need to prove that, $\alpha^{-1}(a_i) = \alpha^{s-1}(a_i)$.
    By the definition of a cycle, we have:
    \[
      \alpha(a_i) = a_{i + 1 \mod s}
    \]
    So, $\alpha^{-1}$ must send $a_{i + 1 \mod s}$ back to $a_i$:
    \[
      \alpha^{-1}(a_{i + 1 \mod s}) = a_i \implies \alpha^{-1}(a_i) = a_{i - 1 \mod s}
    \]
    Apply $\alpha$ $s - 1$ times means moving $a_i$ forward $s - 1$ positions in the cycle.
    Now, we compute $\alpha^{s-1}(a_i)$:
    \begin{align*}
      \alpha^{s-1}(a_i) & = a_{i + (s - 1) \mod s} &  &                                            \\
                        & = a_{i - 1 \mod s}       &  & \text{By } i + (s - 1) \equiv i - 1 \mod s \\
    \end{align*}
    Thus, we have shown that:
    \[
      \alpha^{-1}(a_i) = \alpha^{s-1}(a_i), \forall a_i \in \{a_1, a_2, \cdots , a_s\}
    \]
    For $y \notin \{a_1, a_2, \cdots, a_s\}$, both $\alpha^{-1}(y)$ and $\alpha^{s-1}(y)$ will return $y$ since they do not affect elements outside the cycle.
    Therefore, we conclude that:
    \[
      \alpha^{-1} = \alpha^{s-1}
    \]
  \end{proof}
\end{proposition}


\begin{lemma}
  Let $s \in \Z^+$, and $k \in \Z$.
  Then the sequence $\{0, k, 2k, \ldots, (s - 1)k\} \mod s$ produces all elements of $\Z_s$ if and only if $\gcd(k, s) = 1$.
  \begin{proof}
    We will prove both directions of the statement.
    \begin{itemize}
      \item \textbf{If $\gcd(k, s) = 1$, then the sequence produces all elements of $\Z_s$:}
            Let $S = \{0, k, 2k, \ldots, (s - 1)k\} \mod s$.
            Assume for the sake of contradiction that $S$ does not fully cover $Z_s$.
            Then this means there are repeats in the sequence, i.e., there exist integers $0 \leq i < j < s$ such that:
            \[
              ik \equiv jk \mod s
            \]
            This implies:
            \[
              (j - i)k \equiv 0 \mod s
            \]
            Since $\gcd(k, s) = 1$, then $j - i$ must be a multiple of $s$. Therefore, $j - i = ms$ for some integer $m$, and
            \[
              i \equiv j \mod s
            \]
            This contradicts the assumption that $i < j$.
            Hence, the sequence must cover all elements of $\Z_s$.
      \item \textbf{If the sequence contains all elements of $\Z_s$, then $\gcd(k, s) = 1$:}
            We proceed by contrapositive.
            The contrapositive of the statement is:
            If $\gcd(k, s) > 1$, then the sequence does not produce all elements of $\Z_s$.
            Assume $\gcd(k, s) = d > 1$.
            Then we can write $k = dk'$ and $s = ds'$ for some integers $k'$ and $s'$, and $\gcd(k, s) = d$.
            For $a_i$ in the sequence, we have:
            \[
              a_i = ik \mod s = ik' \cdot d \mod ds' = (ik' \mod s') \cdot d
            \]
            Therefore, $a_i \equiv 0 \mod d$, meaning that all elements in the sequence are multiples of $d$.
            Since $d > 1$, the sequence cannot cover all elements of $\Z_s$ because it only produces multiples of $d$.
            Thus, we conclude that if $\gcd(k, s) > 1$, then the sequence does not produce all elements of $\Z_s$.
    \end{itemize}
    Therefore, we have shown that the sequence $\{0, k, 2k, \ldots, (s - 1)k\} \mod s$ produces all elements of $\Z_s$ if and only if $\gcd(k, s) = 1$.
  \end{proof}
\end{lemma}


\begin{proposition}[Exercise B.4]
  Let $\alpha$ be a cycle of length $s$, say $\alpha = (a_1 a_2 \ldots a_s)$.
  Prove that $\alpha^2$ is a cycle if and only if $s$ is odd.
  \begin{proof}
    We will prove both directions of the statement.
    \begin{itemize}
      \item \textbf{If $\alpha^2$ is a cycle, then $s$ is odd:}
            Assume $\alpha^2$ is a cycle of length $s$.
            Then for $a_i \in \{a_1, a_2, \ldots, a_s\}$, we have:
            \[
              \alpha^2(a_i) = a_{i + 2 \mod s}
            \]
            Apply $\alpha^2$ repeatedly, we get:
            \[
              \alpha^{2k}(a_i) = a_{i + 2k \mod s}
            \]
            Notice that every $a_i$ will eventually return to itself by repeatedly applying $\alpha^2$.
            That is, $a_i = a_{i + 2k \mod s}$, for some integer $k$.
            Define the orbit of $a_i$ under $\alpha^2$ as: the set of all elements that can be reached from $a_i$ by applying $\alpha^2$.
            The orbit length is given by the smallest positive integer $k$ such that:
            \[
              2k \equiv 0 \mod s
            \]
            This implies that $s$ must divide $2k$, that is, for some integer $m$, $2k = sm$.
            Let $d = \gcd(2, s)$, thus one can write $2 = d \cdot 2'$ and $s = d \cdot s'$.
            Therefore,
            \begin{align*}
              2k = sm & \iff (d \cdot 2')k = (d \cdot s')m \\
                      & \iff 2'k = s'm                     \\
            \end{align*}
            Since $2'$ and $s'$ are coprime now, the smallest solution for $k$ is $s'$.
            Therefore,
            \[
              k = \frac{s}{d} = \frac{s}{\gcd(2, s)}
            \]
            For $\alpha^2$ to be a cycle, the length of the orbit should be $s$. Thus,
            \[
              k = s = \frac{s}{\gcd(2, s)} \implies \gcd(2, s) = 1 \implies s \text{ is odd}
            \]
      \item \textbf{If $s$ is odd, then $\alpha^2$ is a cycle:}
            By \textbf{Lemma 4}, since $\gcd(2, s) = 1$, the operation $(+2 \mod s)$ generates all positions in $\Z_s$.
            So every element cycles through all others, implying $\alpha^2$ is a single cycle.
    \end{itemize}
  \end{proof}
\end{proposition}


\begin{proposition}[Exercise B.5]
  Let $\alpha$ be a cycle of length $s$, say $\alpha = (a_1 a_2 \ldots a_s)$.
  Prove that if $s$ is odd, then $\alpha$ is the square of some cycle of length $s$.
  \begin{proof}
    We will first show that $\alpha = \alpha^{s + 1}$.
    \[
      \alpha^{s + 1} = \alpha^s \cdot \alpha = \text{id} \cdot \alpha = \alpha
    \]
    Now, we need to find a cycle $\beta$ such that $\beta^2 = \alpha$.
    I claim that $\beta = \alpha^{\frac{s + 1}{2}}$ works.
    To verify, we compute:
    \[
      \beta^2 = \left(\alpha^{\frac{s + 1}{2}}\right)^2 = \alpha^{\frac{s + 1}{2} \cdot 2} = \alpha^{s + 1} = \alpha
    \]
    Now, we need to check that $\beta$ is indeed a cycle of length $s$.
    By \textbf{Lemma 4}, we know that if $\gcd(k, s) = 1$, then $\alpha^k$ is a cycle of length $s$.
    Therefore, to prove that $\beta$ is a cycle of length $s$, we need to show that $\gcd\left(\frac{s + 1}{2}, s\right) = 1$.
    Since $s$ is odd, we can write $s = 2m  - 1$ for some integer $m$.
    Then,
    \[
      \gcd\left(\frac{s + 1}{2}, s\right) = \gcd\left(\frac{2m}{2}, 2m - 1\right) = \gcd(m, 2m - 1)
    \]
    Assume for the sake of contradiction that $d = \gcd(m, 2m - 1) > 1$.
    Then $d$ divides both $m$ and $2m - 1$, which implies that $d$ divides $2m$.
    Since $d$ also divides $2m - 1$, it must divide their difference:
    \[
      2m - (2m - 1) = 1
    \]
    This is a contradiction since $d$ cannot divide 1, since we assumed that $d > 1$.
    Therefore, we conclude that $\gcd\left(\frac{s + 1}{2}, s\right) = 1$.
    Hence, $\beta = \alpha^{\frac{s + 1}{2}}$ is a cycle of length $s$.
    Thus, we have shown that if $s$ is odd, then $\alpha$ is the square of some cycle of length $s$.
  \end{proof}
\end{proposition}


\begin{proposition}[Exercise B.8]
  Let $\alpha$ be a cycle of length $s$, say $\alpha = (a_1 a_2 \ldots a_s)$.
  Prove that if $s$ is prime, then every power of $\alpha$ is a cycle.
  Formally, show that for every $k \in \Z$ such that $1 \leq k < s$, the cycle $\alpha^k$ is a cycle of length $s$.
  \begin{proof}
    By \textbf{Lemma 4}, we know that if $\gcd(k, s) = 1$, then $\alpha^k$ is a cycle of length $s$.
    Since $s$ is prime, the only divisors of $s$ are 1 and $s$ itself.
    Therefore, for any integer $k$ such that $1 \leq k < s$, we have $\gcd(k, s) = 1$.
    This means that for every $k$ in the range $1 \leq k < s$, $\alpha^k$ is a cycle of length $s$.
  \end{proof}
\end{proposition}


\begin{proposition}[Exercise B.6]
  Let $\alpha$ be a cycle of length $s$, say $\alpha = (a_1 a_2 \ldots a_s)$.
  Prove that if $s$ is even, say $s = 2t$, then $\alpha^2$ is a product of two cycles of length $t$.
  \begin{proof}
    We can express $\alpha$ as:
    \[
      \alpha = (a_1 a_2 \ldots a_{2t})
    \]
    Since by definition $\alpha(a_i) = a_{i + 1 \mod s}$, so we have:
    \[
      \alpha^2(a_i) = a_{i + 2 \mod s} = a_{i + 2 \mod 2t}
    \]
    Let's define two subsets of the cycle $\alpha$:
    \begin{align*}
      A & = \{a_1, a_3, a_5, \ldots, a_{2t - 1}\} \\
      B & = \{a_2, a_4, a_6, \ldots, a_{2t}\}
    \end{align*}
    Notice that $A$ and $B$ partition the set of elements in $\alpha$.
    We can easily see that $\alpha^2$ preserves the elements in these subsets, since add $2$
    to neither the even nor the odd indices will change their parity.
    Now, we need to show that the orbits of $\alpha^2$ on $A$ and $B$ are cycles of length $t$.
    For $a_i \in A$, we have:
    \[
      \alpha^2(a_1) = a_3, \quad \alpha^2(a_3) = a_5, \quad \ldots, \quad \alpha^2(a_{2t - 1}) = a_1
    \]
    This means that applying $\alpha^2$ to $a_i$ will cycle through all elements in $A$, with the orbit as $(a_1a_3\cdots a_{2t-1})$.
    Similarly, for $a_i \in B$, we have the orbit as $(a_2a_4\cdots a_{2t})$. The two orbits are disjoint cycles of length $t$.
    Therefore, we can express $\alpha^2$ as a product of two cycles of length $t$:
    \[
      \alpha^2 = (a_1 a_3 \ldots a_{2t - 1})(a_2 a_4 \ldots a_{2t})
    \]
  \end{proof}
\end{proposition}


% -------------------------------------------------------------- %
%                          Ends here                             %
% -------------------------------------------------------------- %

\end{document}
