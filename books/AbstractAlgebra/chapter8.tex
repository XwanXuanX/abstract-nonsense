\documentclass[12pt]{article}

\usepackage{setspace}
\usepackage[margin=1in]{geometry}
\usepackage{amsmath,amsthm,amssymb,amsfonts}

% Create the margins required for class %
\oddsidemargin=-0.5cm                 	%
\setlength{\textwidth}{6.5in}         	%
\addtolength{\voffset}{-20pt}        		%
\addtolength{\headsep}{25pt}           	%

% Tells LaTeX to allow you to enter information in the heading
\pagestyle{myheadings}
\markright{Yetong Li\hfill \today \hfill}

% Use double spacing throughout the document
% \doublespacing

% Define commonly used sets
\newcommand{\N}{\mathbb{N}} % Natural numbers
\newcommand{\Z}{\mathbb{Z}} % Integers
\newcommand{\R}{\mathbb{R}} % Real numbers
\newcommand{\Q}{\mathbb{Q}} % Rational numbers
\newcommand{\C}{\mathbb{C}} % Complex numbers

% Define a bunch of theorem-like environments
\newtheorem{theorem}{Theorem}
\newtheorem{lemma}{Lemma}
\newtheorem{problem}{Problem}
\newtheorem{question}{Question}
\newtheorem{corollary}{Corollary}
\newtheorem{proposition}{Proposition}

\begin{document}

% -------------------------------------------------------------- %
%                         Starts here                            %
% -------------------------------------------------------------- %

\title{Chapter 8: Permutations of A Finite Set}
\author{Yetong Li \\ A Book of Abstract Algebra}

\maketitle


\begin{lemma}
  For two transpositions, $(xa)$ and $(xb)$, where $a \neq b$,
  prove that $(xa)(xb) = (xa)(ab)$.
  \begin{proof}
    Consider when $z$ is applied to both sides:
    \begin{align*}
      (xa)(xb)(z) & = \begin{cases}
                        z & \text{if } z \neq x, a, b \\
                        a & \text{if } z = x          \\
                        b & \text{if } z = a          \\
                        x & \text{if } z = b
                      \end{cases} \\
    \end{align*}
    On the right-hand side, we have:
    \begin{align*}
      (xa)(ab)(z) & = \begin{cases}
                        z & \text{if } z \neq x, a, b \\
                        a & \text{if } z = x          \\
                        b & \text{if } z = a          \\
                        x & \text{if } z = b
                      \end{cases} \\
    \end{align*}
    Since both sides yield the same result for all $z$, we conclude that:
    \[
      (xa)(xb) = (xa)(ab)
    \]
  \end{proof}
\end{lemma}


\begin{lemma}
  Let $\pi \in S_n$. If $\pi$ is a product of $k$ transpositions, then
  $\pi^{-1}$ is also a product of $k$ transpositions.
  \begin{proof}
    We can express $\pi$ as a product of transpositions:
    \[
      \pi = (x_1 y_1)(x_2 y_2) \cdots (x_k y_k)
    \]
    where each $(x_i y_i)$ is a transposition. The inverse of $\pi$ is given by:
    \[
      \pi^{-1} = (x_k y_k)^{-1} \cdots (x_2 y_2)^{-1} (x_1 y_1)^{-1}
    \]
    Since the inverse of a transposition is itself, we have:
    \[
      \pi^{-1} = (y_k x_k)(y_{k-1} x_{k-1}) \cdots (y_1 x_1)
    \]
    This shows that $\pi^{-1}$ is also a product of $k$ transpositions,
    specifically the same transpositions but in reverse order. Thus, if $\pi$ is a
    product of $k$ transpositions, then $\pi^{-1}$ is also a product of $k$ transpositions.
  \end{proof}
\end{lemma}


\begin{lemma}
  The set of all even permutations in $S_n$ forms a subgroup of $S_n$.
  \begin{proof}
    Let $A_n$ denote the set of all even permutations in $S_n$.
    To show that $A_n$ is a subgroup, we need to show that it is closed under multiplication and inverses.
    \begin{itemize}
      \item \textbf{Closure under multiplication:} If $\pi, \sigma \in A_n$,
            then both $\pi$ and $\sigma$ can be expressed as products of an even number of transpositions.
            The product $\pi \sigma$ can be expressed as a product of the transpositions of $\pi$ followed by those of $\sigma$.
            Since the sum of two even numbers is even, $\pi \sigma$ is also a product of an even number of transpositions, hence $\pi \sigma \in A_n$.
      \item \textbf{Closure under inverses:} If $\pi \in A_n$, then $\pi$ can be expressed as a product of an even number of transpositions.
            By \textbf{Lemma 2}, the inverse $\pi^{-1}$ is also a product of the same transpositions in reverse order.
            Since the number of transpositions remains even, $\pi^{-1} \in A_n$.
    \end{itemize}
    Since $A_n$ is closed under multiplication and inverses, it is a subgroup of $S_n$.
  \end{proof}
\end{lemma}


% -------------------------------------------------------------- %
%                          Ends here                             %
% -------------------------------------------------------------- %

\end{document}
