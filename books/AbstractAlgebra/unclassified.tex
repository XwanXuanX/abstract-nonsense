\documentclass[12pt]{article}

\usepackage{setspace}
\usepackage[margin=1in]{geometry}
\usepackage{amsmath,amsthm,amssymb,amsfonts}

% Create the margins required for class %
\oddsidemargin=-0.5cm                 	%
\setlength{\textwidth}{6.5in}         	%
\addtolength{\voffset}{-20pt}        		%
\addtolength{\headsep}{25pt}           	%

% Tells LaTeX to allow you to enter information in the heading
\pagestyle{myheadings}
\markright{Yetong Li\hfill \today \hfill}

% Use double spacing throughout the document
% \doublespacing

% Define commonly used sets
\newcommand{\N}{\mathbb{N}} % Natural numbers
\newcommand{\Z}{\mathbb{Z}} % Integers
\newcommand{\R}{\mathbb{R}} % Real numbers
\newcommand{\Q}{\mathbb{Q}} % Rational numbers
\newcommand{\C}{\mathbb{C}} % Complex numbers

% Define a bunch of theorem-like environments
\newtheorem{theorem}{Theorem}
\newtheorem{lemma}{Lemma}
\newtheorem{problem}{Problem}
\newtheorem{question}{Question}
\newtheorem{corollary}{Corollary}
\newtheorem{proposition}{Proposition}

\begin{document}

% -------------------------------------------------------------- %
%                         Starts here                            %
% -------------------------------------------------------------- %

\title{Unclassified Proofs}
\author{Yetong Li \\ A Book of Abstract Algebra}

\maketitle


\begin{proposition}
  In any finite group $G$, the number of elements not equal to their own inverse is even.
  \begin{proof}
    Let $S = \{ x \in G \mid x \neq x^{-1} \}$.
    For each $x \in S$, its inverse $x^{-1}$ is also in $S$, and $x \neq x^{-1}$.
    Thus, $S$ can be partitioned into distinct pairs $(x, x^{-1})$.
    The number of elements in $S$ is therefore even, since every element appears in exactly one pair.
  \end{proof}
\end{proposition}


\begin{proposition}
  In any finite group $G$, the number of elements in $G$ that are equal to its own inverse is odd or even,
  depending on whether $|G|$ is odd or even.
  \begin{proof}
    Let $A = \{x \in G : x \neq x^{-1}\}$ and $B = \{x \in G : x = x^{-1}\}$.
    Note that $A \cap B = \emptyset$ and $A \cup B = G$. This means that $|G| = |A| + |B|$.
    From the previous proposition, we know that $|A|$ is even.
    Therefore, if $|G|$ is odd, then $|B|$ must also be odd, and if $|G|$ is even, then $|B|$ must also be even.
    Thus, the number of elements in $G$ that are equal to its own inverse is odd if and only if $|G|$ is odd.
  \end{proof}
\end{proposition}


\begin{proposition}
  Prove that $(bab^{-1})^n = ba^nb^{-1}$, for all $n \in \N$.
  \begin{proof}
    We will prove this by induction on $n$.
    \begin{itemize}
      \item Base case: For $n = 1$, we have $(bab^{-1})^1 = bab^{-1} = ba^1b^{-1}$, which holds true.
      \item Inductive step: Assume it holds for some $n = k$, i.e., $(bab^{-1})^k = ba^kb^{-1}$.
            We need to show it holds for $n = k + 1$:
            \begin{align*}
              (bab^{-1})^{k+1} & = (bab^{-1})^k \cdot (bab^{-1}) \\
                               & = (ba^kb^{-1})(bab^{-1})        \\
                               & = ba^kb^{-1}bab^{-1}            \\
                               & = ba^k(b^{-1}b)ab^{-1}          \\
                               & = ba^{k+1}b^{-1}
            \end{align*}
            Thus, the inductive step holds.
    \end{itemize}
    By the principle of mathematical induction, we conclude that $(bab^{-1})^n = ba^nb^{-1}$ for all $n \in \N$.
  \end{proof}
\end{proposition}


\begin{proposition}
  Let $G$ be an abelian group. Let $H = \{x \in G : x^n = e\}$. Prove that $H \leqslant G$.
  \begin{proof}
    It's clear that $H \subseteq G$. We need to show that $H$ is closed under multiplication and closed under inverses.
    \begin{itemize}
      \item Closure under multiplication: Let $x, y \in H$. Then $x^n = e$ and $y^n = e$.
            We need to show that $(xy)^n = e$:
            \begin{align*}
              (xy)^n & = x^n y^n &  & \text{[By abelian]} \\
                     & = ee                               \\
                     & = e.
            \end{align*}
            Thus, $xy \in H$.
      \item Closure under inverses: Let $x \in H$. Then $x^n = e$.
            We need to show that $x^{-1} \in H$:
            \begin{align*}
              (x^{-1})^n & = (x^n)^{-1} &  & \text{[By the property of inverses]} \\
                         & = e^{-1}                                               \\
                         & = e.
            \end{align*}
            Thus, $x^{-1} \in H$.
    \end{itemize}
    Since $H$ is closed under multiplication and inverses, we conclude that $H$ is a subgroup of $G$.
  \end{proof}
\end{proposition}


\begin{proposition}
  Let $G$ be an abelian group. Let $H = \{x \in G : \exists y \in G, x = y^2\}$. Prove that $H \leqslant G$.
  \begin{proof}
    It's clear that $H \subseteq G$. We need to show that $H$ is closed under multiplication and closed under inverses.
    \begin{itemize}
      \item Closure under multiplication: Let $x, y \in H$. Then there exist $a, b \in G$ such that $x = a^2$ and $y = b^2$.
            We need to show that $xy \in H$:
            \begin{align*}
              xy & = a^2b^2 &  & \text{[By the definition of } x \text{ and } y]          \\
                 & = (ab)^2 &  & \text{[By the property of exponents in abelian groups]}.
            \end{align*}
            Thus, $xy \in H$.
      \item Closure under inverses: Let $x \in H$. Then there exists $y \in G$ such that $x = y^2$.
            We need to show that $x^{-1} \in H$:
            \begin{align*}
              x^{-1} & = (y^2)^{-1} &  & \text{[By the property of inverses]}                     \\
                     & = (y^{-1})^2 &  & \text{[By the property of exponents in abelian groups]}.
            \end{align*}
            Thus, $x^{-1} \in H$.
    \end{itemize}
    Since $H$ is closed under multiplication and inverses, we conclude that $H$ is a subgroup of $G$.
  \end{proof}
\end{proposition}


\begin{proposition}
  Let $G$ be an abelian group. Let $H \leqslant G$. Let $K = \{x \in G : x^2 \in H\}$.
  Prove that $K \leqslant G$.
  \begin{proof}
    It's clear that $K \subseteq G$. We need to show that $K$ is closed under multiplication and closed under inverses.
    \begin{itemize}
      \item Closure under multiplication: Let $x, y \in K$. Then $x^2 \in H$ and $y^2 \in H$.
            We need to show that $(xy)^2 \in H$:
            \begin{align*}
              (xy)^2 & = x^2y^2 &  & \text{[By the property of exponents in abelian groups]}                  \\
                     & \in H    &  & \text{[Since } H \text{ is a subgroup and closed under multiplication]}.
            \end{align*}
            Thus, $xy \in K$.
      \item Closure under inverses: Let $x \in K$. Then $x^2 \in H$.
            We need to show that $(x^{-1})^2 \in H$:
            \begin{align*}
              (x^{-1})^2 & = (x^2)^{-1} &  & \text{[By the property of inverses]}
            \end{align*}
            Since $H$ is a subgroup, it is closed under inverses, so $(x^2)^{-1} \in H$.
            Thus, $x^{-1} \in K$.
    \end{itemize}
    Since $K$ is closed under multiplication and inverses, we conclude that $K$ is a subgroup of $G$.
  \end{proof}
\end{proposition}


\begin{proposition}
  Let $H \leqslant G$ and $K \leqslant G$.
  Prove that $H \cap K \leqslant G$.
  \begin{proof}
    Since $H \subseteq G$ and $K \subseteq G$, it follows that $H \cap K \subseteq G$.
    We need to show that $H \cap K$ is closed under multiplication and closed under inverses.
    \begin{itemize}
      \item  Closure under multiplication: Let $x, y \in H \cap K$.
            Then $x \in H$, $y \in H$, $x \in K$, and $y \in K$.
            Since $H$ and $K$ are subgroups, we have $xy \in H$ and $xy \in K$.
            Therefore, $xy \in H \cap K$.
      \item Closure under inverses: Let $x \in H \cap K$.
            Then $x \in H$ and $x \in K$.
            Since $H$ and $K$ are subgroups, we have $x^{-1} \in H$ and $x^{-1} \in K$.
            Therefore, $x^{-1} \in H \cap K$.
    \end{itemize}
    Since $H \cap K$ is closed under multiplication and inverses, we conclude that $H \cap K$ is a subgroup of $G$.
  \end{proof}
\end{proposition}


\begin{proposition}
  Let $H \leqslant G$ and $K \leqslant G$.
  Prove that if $H \subseteq K$, then $H \leqslant K$.
  \begin{proof}
    Since $K \leqslant G$, then $K$ is a group.
    Since $H \subseteq G$, $H \neq \emptyset$ and $H$ is closed under multiplication and inverses.
    Since $H \subseteq K$, then $H$ is a non-empty subset of $K$.
    By the definition of a subgroup, we proved that $H \leqslant K$.
  \end{proof}
\end{proposition}


\begin{proposition}
  Let $G$ be a finite group. Let $S$ be a non-empty subset of $G$, and $S$ is closed under multiplication.
  Prove that $S \leqslant G$.
  \begin{proof}
    To show that $S$ is a subgroup of $G$, we need to show that $S$ contains the identity element and is closed under inverses.
    \begin{itemize}
      \item Identity element: Let $S = \{a_1, a_2, \ldots, a_n\}$. If $a_i \in S$, then the $n$ products
            $A = \{a_ia_1, a_ia_2, \ldots, a_ia_n\}$ are elements of $S$ since $S$ is closed under multiplication.
            Note that all the $n$ products are distinct. Assume that they are not distinct, that is, $\exists a_j, a_k \in S$ such that
            $a_j \neq a_k$ and $a_ia_j = a_ia_k$. Then we would have $a_j = a_k$, which contradicts the assumption that $a_j \neq a_k$.
            Therefore, $A$ contains $n$ distinct elements, and $|S| = |A|$. Therefore, each element of $S$ must correspond to one element in $A$.
            Specifically, $a_1 = a_1a_k$, for some $k \in \{1, 2, \ldots, n\}$. Then, we have
            \begin{align*}
              a_1 & = a_1e = a_1a_k \\
              e   & = a_k
            \end{align*}
            Thus, $e \in S$.
      \item Closure under inverses: For all $a_i \in S$, let $A$ be the set of $n$ products. As we showed before, that $|S| = |A| = n$.
            And since $e \in S$, this means $\exists a_j \in S$ such that $a_ia_j = e$.
            Thus, $a_i$ and $a_j$ are inverses of each other.
            Since we can always find an inverse for all $a_i \in S$. We proved that $S$ is closed under inverses.
    \end{itemize}
    Since $S$ contains the identity element and is closed under inverses, we conclude that $S$ is a subgroup of $G$.
  \end{proof}
\end{proposition}


\begin{proposition}
  Let $G$ be a group and $f: G \to G$. Prove that the set of all the periods of $f$
  is a subgroup of $G$.
  \begin{proof}
    Let $S = \{a \in G : f(x) = f(ax), \forall x \in G\}$.
    We need to show that $S$ is non-empty subset, closed under multiplication, and closed under inverses.
    \begin{itemize}
      \item Non-empty subset:
            Let $a = e$, then $f(x) = f(ex), \forall x \in G$.
            Thus $S \neq \emptyset$. Since $\forall a \in S, a \in G$, then $S \subseteq G$.
      \item Closure under multiplication:
            Let $a, b \in S$.
            Then $f(x) = f(ax), \forall x \in G$ and $f(y) = f(by), \forall y \in G$.
            In particular, let $x = by$ and $by \in G$.
            Then we have
            \begin{align*}
              f(by) & = f(a(by)) \\
                    & = f((ab)y) \\
                    & = f(y)
            \end{align*}
            Therefore, we've shown that $f((ab)y) = f(y), \forall y \in G$.
            Thus, $ab \in S$.
      \item Closure under inverses:
            Let $a \in S$ such that $f(x) = f(ax), \forall x \in G$.
            Let $x = a^{-1}a^{-1}x$ and note that $a^{-1}a^{-1}x \in G$.
            Then we have
            \begin{align*}
              f(ax) & = f(aa^{-1}a^{-1}x) \\
                    & = f(a^{-1}x)        \\
                    & = f(x).
            \end{align*}
            Therefore, $f(x) = f(a^{-1}x), \forall x \in G$.
            Thus, $a^{-1} \in S$.
    \end{itemize}
    Since $S$ is non-empty, closed under multiplication, and closed under inverses, we conclude that $S$ is a subgroup of $G$.
  \end{proof}
\end{proposition}


% -------------------------------------------------------------- %
%                          Ends here                             %
% -------------------------------------------------------------- %

\end{document}
