\documentclass[12pt]{article}

\usepackage{setspace}
\usepackage[margin=1in]{geometry}
\usepackage{amsmath,amsthm,amssymb,amsfonts}

% Create the margins required for class %
\oddsidemargin=-0.5cm                 	%
\setlength{\textwidth}{6.5in}         	%
\addtolength{\voffset}{-20pt}        		%
\addtolength{\headsep}{25pt}           	%

% Tells LaTeX to allow you to enter information in the heading
\pagestyle{myheadings}
\markright{Yetong Li\hfill \today \hfill}

% Use double spacing throughout the document
% \doublespacing

% Define commonly used sets
\newcommand{\N}{\mathbb{N}} % Natural numbers
\newcommand{\Z}{\mathbb{Z}} % Integers
\newcommand{\R}{\mathbb{R}} % Real numbers
\newcommand{\Q}{\mathbb{Q}} % Rational numbers
\newcommand{\C}{\mathbb{C}} % Complex numbers

% Define a bunch of theorem-like environments
\newtheorem{theorem}{Theorem}
\newtheorem{lemma}{Lemma}
\newtheorem{problem}{Problem}
\newtheorem{question}{Question}
\newtheorem{corollary}{Corollary}
\newtheorem{proposition}{Proposition}

\begin{document}

% -------------------------------------------------------------- %
%                         Starts here                            %
% -------------------------------------------------------------- %

\title{Unclassified Proofs}
\author{Yetong Li \\ A Book of Abstract Algebra}

\maketitle


\begin{proposition}
  In any finite group $G$, the number of elements not equal to their own inverse is even.
  \begin{proof}
    Let $S = \{ x \in G \mid x \neq x^{-1} \}$.
    For each $x \in S$, its inverse $x^{-1}$ is also in $S$, and $x \neq x^{-1}$.
    Thus, $S$ can be partitioned into distinct pairs $(x, x^{-1})$.
    The number of elements in $S$ is therefore even, since every element appears in exactly one pair.
  \end{proof}
\end{proposition}


\begin{proposition}
  In any finite group $G$, the number of elements in $G$ that are equal to its own inverse is odd or even,
  depending on whether $|G|$ is odd or even.
  \begin{proof}
    Let $A = \{x \in G : x \neq x^{-1}\}$ and $B = \{x \in G : x = x^{-1}\}$.
    Note that $A \cap B = \emptyset$ and $A \cup B = G$. This means that $|G| = |A| + |B|$.
    From the previous proposition, we know that $|A|$ is even.
    Therefore, if $|G|$ is odd, then $|B|$ must also be odd, and if $|G|$ is even, then $|B|$ must also be even.
    Thus, the number of elements in $G$ that are equal to its own inverse is odd if and only if $|G|$ is odd.
  \end{proof}
\end{proposition}


\begin{proposition}
  Prove that $(bab^{-1})^n = ba^nb^{-1}$, for all $n \in \N$.
  \begin{proof}
    We will prove this by induction on $n$.
    \begin{itemize}
      \item Base case: For $n = 1$, we have $(bab^{-1})^1 = bab^{-1} = ba^1b^{-1}$, which holds true.
      \item Inductive step: Assume it holds for some $n = k$, i.e., $(bab^{-1})^k = ba^kb^{-1}$.
            We need to show it holds for $n = k + 1$:
            \begin{align*}
              (bab^{-1})^{k+1} & = (bab^{-1})^k \cdot (bab^{-1}) \\
                               & = (ba^kb^{-1})(bab^{-1})        \\
                               & = ba^kb^{-1}bab^{-1}            \\
                               & = ba^k(b^{-1}b)ab^{-1}          \\
                               & = ba^{k+1}b^{-1}
            \end{align*}
            Thus, the inductive step holds.
    \end{itemize}
    By the principle of mathematical induction, we conclude that $(bab^{-1})^n = ba^nb^{-1}$ for all $n \in \N$.
  \end{proof}
\end{proposition}


\begin{proposition}
  Let $G$ be an abelian group. Let $H = \{x \in G : x^n = e\}$. Prove that $H \leqslant G$.
  \begin{proof}
    It's clear that $H \subseteq G$. We need to show that $H$ is closed under multiplication and closed under inverses.
    \begin{itemize}
      \item Closure under multiplication: Let $x, y \in H$. Then $x^n = e$ and $y^n = e$.
            We need to show that $(xy)^n = e$:
            \begin{align*}
              (xy)^n & = x^n y^n &  & \text{[By abelian]} \\
                     & = ee                               \\
                     & = e.
            \end{align*}
            Thus, $xy \in H$.
      \item Closure under inverses: Let $x \in H$. Then $x^n = e$.
            We need to show that $x^{-1} \in H$:
            \begin{align*}
              (x^{-1})^n & = (x^n)^{-1} &  & \text{[By the property of inverses]} \\
                         & = e^{-1}                                               \\
                         & = e.
            \end{align*}
            Thus, $x^{-1} \in H$.
    \end{itemize}
    Since $H$ is closed under multiplication and inverses, we conclude that $H$ is a subgroup of $G$.
  \end{proof}
\end{proposition}


\begin{proposition}
  Let $G$ be an abelian group. Let $H = \{x \in G : \exists y \in G, x = y^2\}$. Prove that $H \leqslant G$.
  \begin{proof}
    It's clear that $H \subseteq G$. We need to show that $H$ is closed under multiplication and closed under inverses.
    \begin{itemize}
      \item Closure under multiplication: Let $x, y \in H$. Then there exist $a, b \in G$ such that $x = a^2$ and $y = b^2$.
            We need to show that $xy \in H$:
            \begin{align*}
              xy & = a^2b^2 &  & \text{[By the definition of } x \text{ and } y]          \\
                 & = (ab)^2 &  & \text{[By the property of exponents in abelian groups]}.
            \end{align*}
            Thus, $xy \in H$.
      \item Closure under inverses: Let $x \in H$. Then there exists $y \in G$ such that $x = y^2$.
            We need to show that $x^{-1} \in H$:
            \begin{align*}
              x^{-1} & = (y^2)^{-1} &  & \text{[By the property of inverses]}                     \\
                     & = (y^{-1})^2 &  & \text{[By the property of exponents in abelian groups]}.
            \end{align*}
            Thus, $x^{-1} \in H$.
    \end{itemize}
    Since $H$ is closed under multiplication and inverses, we conclude that $H$ is a subgroup of $G$.
  \end{proof}
\end{proposition}


% -------------------------------------------------------------- %
%                          Ends here                             %
% -------------------------------------------------------------- %

\end{document}
