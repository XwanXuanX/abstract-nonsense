\documentclass[12pt]{article}

\usepackage{setspace}
\usepackage[margin=1in]{geometry}
\usepackage{amsmath,amsthm,amssymb,amsfonts}

% Create the margins required for class %
\oddsidemargin=-0.5cm                 	%
\setlength{\textwidth}{6.5in}         	%
\addtolength{\voffset}{-20pt}        		%
\addtolength{\headsep}{25pt}           	%

% Tells LaTeX to allow you to enter information in the heading
\pagestyle{myheadings}
\markright{Yetong Li\hfill \today \hfill}

% Use double spacing throughout the document
% \doublespacing

% Define commonly used sets
\newcommand{\N}{\mathbb{N}} % Natural numbers
\newcommand{\Z}{\mathbb{Z}} % Integers
\newcommand{\R}{\mathbb{R}} % Real numbers
\newcommand{\Q}{\mathbb{Q}} % Rational numbers
\newcommand{\C}{\mathbb{C}} % Complex numbers

% Define a bunch of theorem-like environments
\newtheorem{theorem}{Theorem}
\newtheorem{lemma}{Lemma}
\newtheorem{problem}{Problem}
\newtheorem{question}{Question}
\newtheorem{corollary}{Corollary}
\newtheorem{proposition}{Proposition}

\begin{document}

% -------------------------------------------------------------- %
%                         Starts here                            %
% -------------------------------------------------------------- %

\title{Homework LaTeX Template}
\author{Yetong Li \\ Foundations of Mathematics} % replace with the course title

\maketitle


\begin{proposition}
  Let $n \in \mathbb{N}$. Prove that $\sum_{i=1}^n i = \frac{n(n+1)}{2}$.
  \begin{proof}
    We proceed by induction on $n$. \\
    \textbf{Base case:} For $n=1$, $\sum_{i=1}^1 i = 1 = \frac{1 \cdot 2}{2}$. \\
    \textbf{Inductive step:} Assume true for $n=k$: $\sum_{i=1}^k i = \frac{k(k+1)}{2}$. \\
    For $n=k+1$:
    \begin{align*}
      \sum_{i=1}^{k+1} i & = \left(\sum_{i=1}^k i\right) + (k+1) \\
                         & = \frac{k(k+1)}{2} + (k+1)            \\
                         & = \frac{k(k+1) + 2(k+1)}{2}           \\
                         & = \frac{(k+1)(k+2)}{2}
    \end{align*}
    Thus, the result holds for all $n \in \mathbb{N}$.
  \end{proof}
\end{proposition}


% -------------------------------------------------------------- %
%                          Ends here                             %
% -------------------------------------------------------------- %

\end{document}
