\documentclass[12pt]{article}

\usepackage{setspace}
\usepackage[margin=1in]{geometry}
\usepackage{amsmath,amsthm,amssymb,amsfonts}

% Create the margins required for class %
\oddsidemargin=-0.5cm                 	%
\setlength{\textwidth}{6.5in}         	%
\addtolength{\voffset}{-20pt}        		%
\addtolength{\headsep}{25pt}           	%

% Tells LaTeX to allow you to enter information in the heading
\pagestyle{myheadings}
\markright{Yetong Li\hfill \today \hfill}

% Use double spacing throughout the document
% \doublespacing

% Define commonly used sets
\newcommand{\N}{\mathbb{N}} % Natural numbers
\newcommand{\Z}{\mathbb{Z}} % Integers
\newcommand{\R}{\mathbb{R}} % Real numbers
\newcommand{\Q}{\mathbb{Q}} % Rational numbers
\newcommand{\C}{\mathbb{C}} % Complex numbers

% Define a bunch of theorem-like environments
\newtheorem{theorem}{Theorem}
\newtheorem{lemma}{Lemma}
\newtheorem{problem}{Problem}
\newtheorem{question}{Question}
\newtheorem{corollary}{Corollary}
\newtheorem{proposition}{Proposition}

\begin{document}

% -------------------------------------------------------------- %
%                         Starts here                            %
% -------------------------------------------------------------- %

\title{Chapter 4: Proof Exercises}
\author{Yetong Li \\ Proofs: Long Form Mathematical Textbook}

\maketitle


\begin{proposition}
  Suppose $A_1, A_2, \ldots, A_n$ are subsets of a set $U$.
  Prove that the following holds for every $n \in \N$.
  \[
    \left(\bigcap_{i=1}^n A_i\right)^c = \bigcup_{i=1}^n A_i^c
  \]
  \begin{proof}
    First, we use the fact that unions and intersections are associative.
    Thus,
    \[
      A_1 \cap A_2 \cap \ldots \cap A_n = (((A_1 \cap A_2) \cap A_3) \cap \ldots) \cap A_n
    \]
    Then we proceed by induction on $n$.
    \begin{itemize}
      \item Base case: $n = 2$. $(A_1 \cap A_2)^c = A_1^c \cup A_2^c$ holds by De Morgan's laws.
      \item Inductive step: Assume the statement holds for $n = k$.
            That is, assume $(A_1 \cap A_2 \cap \ldots \cap A_k)^c = A_1^c \cup A_2^c \cup \ldots \cup A_k^c$.
            Then for $n = k + 1$,
            \begin{align*}
              (A_1 \cap A_2 \cap \ldots \cap A_k \cap A_{k+1})^c & =
              ((A_1 \cap A_2 \cap \ldots \cap A_k) \cap A_{k+1})^c                                                            \\
                                                                 & = (A_1 \cap A_2 \cap \ldots \cap A_k)^c \cup A_{k+1}^c     \\
                                                                 & = (A_1^c \cup A_2^c \cup \ldots \cup A_k^c) \cup A_{k+1}^c \\
                                                                 & = A_1^c \cup A_2^c \cup \ldots \cup A_k^c \cup A_{k+1}^c   \\
                                                                 & = \bigcup_{i=1}^{k+1} A_i^c
            \end{align*}
    \end{itemize}
    By the principle of mathematical induction, the statement holds for all $n \in \N$.
  \end{proof}
\end{proposition}


% -------------------------------------------------------------- %
%                          Ends here                             %
% -------------------------------------------------------------- %

\end{document}
